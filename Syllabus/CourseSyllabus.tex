\documentclass[11pt, a4paper]{article}
%\usepackage{geometry}
\usepackage[inner=1.5cm,outer=1.5cm,top=2.5cm,bottom=2.5cm]{geometry}
\pagestyle{empty}
\usepackage{graphicx}
\usepackage{fancyhdr, lastpage, bbding, pmboxdraw}
\usepackage[usenames,dvipsnames]{color}
\definecolor{darkblue}{rgb}{0,0,.6}
\definecolor{darkred}{rgb}{.7,0,0}
\definecolor{darkgreen}{rgb}{0,.6,0}
\definecolor{red}{rgb}{.98,0,0}
\usepackage[colorlinks,pagebackref,pdfusetitle,urlcolor=darkblue,citecolor=darkblue,linkcolor=darkred,bookmarksnumbered,plainpages=false]{hyperref}
\renewcommand{\thefootnote}{\fnsymbol{footnote}}

\pagestyle{fancyplain}
\fancyhf{}
\lhead{ \fancyplain{}{NRES 779, Spring 2026} }
%\chead{ \fancyplain{}{} }
\rhead{ \fancyplain{}{\today} }
%\rfoot{\fancyplain{}{page \thepage\ of \pageref{LastPage}}}
\fancyfoot[RO, LE] {page \thepage\ of \pageref{LastPage} }
\thispagestyle{plain}

%%%%%%%%%%%% LISTING %%%
\usepackage{listings}
\usepackage{caption}
\DeclareCaptionFont{white}{\color{white}}
\DeclareCaptionFormat{listing}{\colorbox{gray}{\parbox{\textwidth}{#1#2#3}}}
\captionsetup[lstlisting]{format=listing,labelfont=white,textfont=white}
\usepackage{verbatim} % used to display code
\usepackage{fancyvrb}
\usepackage{acronym}
\usepackage{amsthm}
\VerbatimFootnotes % Required, otherwise verbatim does not work in footnotes!



\definecolor{OliveGreen}{cmyk}{0.64,0,0.95,0.40}
\definecolor{CadetBlue}{cmyk}{0.62,0.57,0.23,0}
\definecolor{lightlightgray}{gray}{0.93}



\lstset{
%language=bash,                          % Code langugage
basicstyle=\ttfamily,                   % Code font, Examples: \footnotesize, \ttfamily
keywordstyle=\color{OliveGreen},        % Keywords font ('*' = uppercase)
commentstyle=\color{gray},              % Comments font
numbers=left,                           % Line nums position
numberstyle=\tiny,                      % Line-numbers fonts
stepnumber=1,                           % Step between two line-numbers
numbersep=5pt,                          % How far are line-numbers from code
backgroundcolor=\color{lightlightgray}, % Choose background color
frame=none,                             % A frame around the code
tabsize=2,                              % Default tab size
captionpos=t,                           % Caption-position = bottom
breaklines=true,                        % Automatic line breaking?
breakatwhitespace=false,                % Automatic breaks only at whitespace?
showspaces=false,                       % Dont make spaces visible
showtabs=false,                         % Dont make tabls visible
columns=flexible,                       % Column format
morekeywords={__global__, __device__},  % CUDA specific keywords
}

%%%%%%%%%%%%%%%%%%%%%%%%%%%%%%%%%%%%
\begin{document}
\begin{center} {\Large \textsc{Bayesian Hierarchical Modeling in
      Natural Resources (NRES 779)}}
\end{center}
\begin{center}
  Spring 2024
\end{center}

%%%%%%%%%%%%%%%%%%%%%%%%%%%%%%%%%

\section*{Course Information}
\subsection*{Meeting Times}
\textbf{Class} (3 credits): Wednesday and  Friday 9-9:50 (Orvis Building OB202), \\
\textbf{Labs:} Friday 10:00-12:00 (Fleischmann Ag FA234)

%%%

\subsection*{Instructor Information:}
\textbf{Instructor:} Perry Williams, Ph.D. \\
\textbf{Office:} FA240\\
\textbf{Email:} perryw@unr.edu\\
\textbf{Website:} www.perrywilliams.us \\
\textbf{Student/Instructor Communication} Email (I do not check messages on WebCampus) \\
\textbf{Office Hours:} After class or by appointment

%%%

\subsection*{Course Description:}
Virtually all progress is science requires using models to gain
insight from data. This course will focus on gaining insight of
scientific processes using statistics, mathematics, and
observation. We will first review necessary probability, statistics,
and computing, and then focus on Bayesian model building and
implementation. The review at the beginning is primarily meant to get
everyone on the same page in terms of notation and to highlight the
most important aspects of the prerequisites. The data used in
applications will focus on ecological applications. However, many of
the concepts and models are transferable to other areas of natural
resources and environmental science, including evolution, and
conservation biology. Example data sets will include: wildlife and
plant surveys, presence-absence (occupancy) data,
biodiversity data, and physical
and chemical process data. We will strive to discuss methods that are generally
applicable to all areas of natural resource and environmental
science. Students should have access to their own computer and a
working installation of R.

\subsection*{Course prerequisites}
\subsubsection*{Required prerequisites}
\begin{itemize}
\item Working knowledge of \texttt{R}
\end{itemize}

\subsubsection*{Suggested prerequisites}
Although not necessary, students with a course in calculus, linear
algebra (MATH 330), and one of the following
courses
\begin{itemize}
\item NRES 488,
\item STAT 429/629,
\item STAT 461/661,
\item NRES 710,
\item NRES 746,
\end{itemize}
will get the most out of the course. None of these are absolute
requirements; I will review key background concepts as part of the
lectures. However, that said, if you don't have at least two of these
background courses, you should be prepared to do some remedial work on
your own.

\subsection*{Required texts:}
\begin{itemize}
\item Hobbs, N. T. and M. B. Hooten, Bayesian models: A statistical primer
for ecologists, Princeton University Press 2015.
\end{itemize}

\subsection*{Additional Reading Material:}
\begin{itemize}
\item Assigned periodically
\end{itemize}


\subsection*{Lecture Schedule}
\begin{itemize}
\item 1/21 Lecture 01: Introduction
\item 1/26 Lecture 02: What sets Bayes apart
\item 1/28 Lecture 03: Deterministic models
\item 2/2 Lecture 04: Linking models to data
\item 2/4 Lecture 05: Rules of probability
\item 2/9 Lecture 06: Probability distributions 
\item 2/11 Lecture 07: Moment matching
\item 2/16 Lecture 08: Likelihood
\item 2/18 Lecture 09: Maximum Likelihood
\item 2/23 Lecture 10: Bayes' theorem
\item 2/25 Lecture 11: Priors
\item 3/2 Lecture 12: Conjugate Priors
\item 3/4 Lecture 13: MCMC I
\item 3/9 Lecture 14: MCMC II
\item 3/11 Lecture 15: MCMC III
\item 3/16 Lecture 16: MCMC IV
\item 3/18 Lecture 17: Review/Applications
\item Spring Break
\item 3/30 Lecture 18: Application: Bayesian Hierarchical Modeling I
\item 4/1 Lecture 19: Application: Bayesian Hierarchical Modeling I continued
\item 4/6 Lecture 20: Bayesian Model Checking 
\item 4/8 Lecture 21: Bayesian Model Selection
\item 4/13 Lecture 22: Application: Bayesian Hierarchical Modeling II
\item 4/15 Lecture 23: Application: Bayesian Hierarchical Modeling II continued
\item  4/21 Lecture 24:  Application: Bayesian Hierarchical Modeling III
\item 4/22 Lecture 25:  Application: Bayesian Hierarchical Modeling III continued
\item 4/27 Lecture 26:  Application: Bayesian Hierarchical Modeling IV
\item 4/29 Lecture 27:  Application: Bayesian Hierarchical Modeling IV continued
\item 5/4 Lecture 28:  Application: Bayesian Hierarchical Modeling V
\item 5/9 Lecture 29:  Application: Bayesian Hierarchical Modeling V continued
\end{itemize}

\subsection*{Lab Schedule}
\begin{itemize}
\item 1/21 Lab 1 Assigned (Programming in R) 
\item 1/23 Lab 1 Workday
\item 2/4 Lab 1 Due

\item 2/4 Lab 2 Assigned (Probability)
\item 2/6 Lab 2 Workday
\item 2/14 Lab 2 Workday
\item 2/16 Lab 2 Due

\item 2/16 Lab 3 Assigned (Likelihood)
\item 2/20 Lab 3 Workday 
\item 2/23 Lab 3 Due

\item 2/23 Lab 4 Assigned (Bayes Theorem)
\item 2/27 Lab 4 Workday
\item 3/2 Lab 4 Due

\item 3/2 Lab 5 Assigned (MCMC)
\item 3/6 Lab 5 Workday
\item 3/11 Lab 5 Due

\item 3/11 Lab 6 Assigned (Metropolis-Hastings)
\item 3/13 Lab 6 Workday
\item 3/20 Lab 6 Due

\item Spring Break

\item 3/30 Lab 7 Assigned (Application I in R and Jags)
\item 4/3 Lab 7 Workday
\item 4/10 Lab 7 Workday
\item 4/13 Lab 7 Due

\item 4/13 Lab 8 Assigned (Application II)
\item 4/17 Lab 8 Workday
\item 4/20 Lab 8 Due

\item 4/20 Lab 9 Assigned (Application III)
\item 4/24 Lab 9 Workday
\item 4/27 Lab 9 Due
\item 4/27 Lab 10 Assigned (Application IV)
\item 5/1 Lab 10 Workday
\item 5/8 Lab 10 Workday
\item 5/13 Lab 10 Due









\end{itemize}







%%%%%%%%%%%%%%%%%%%%%%%%%%%%%%%%%%%%%%%%%%%%%%%%%%%%

\section*{Student learning outcomes:}
\textbf{Students will be able to:}
\begin{enumerate}
\item Learn the basic principles of probability and statistical
  distributions needed to link deterministic models to data and apply these to a number of real data sets.
\item Explain maximum likelihood methods for estimating parameters in
  ecological models.
\item Explain key principles of Bayesian statistics. Understand the
  relationship between inference accomplished by maximum likelihood
  and by applying Bayes’ theorem.
\item Diagram, write, and implement hierarchical models
  appropriate for diverse problems in ecological and natural resource science.
\item Explain how Markov chain Monte Carlo (MCMC) methods can be used
  to approximate marginal posterior distributions. Write MCMC
  algorithms and computer code in \texttt{R} implementing MCMC methods
  for simple Bayesian models.
\item Use software for implementing MCMC methods (i.e., JAGS, R
  packages) to approximate marginal posterior distributions of
  parameters, latent variables, and derived quantities of interest. Be
  able to evaluate convergence.
\item Apply procedures for model checking and model selection in
  the Bayesian framework.
\end{enumerate}

%%%%%%%%%%%%%%%%%%%%%%%%%%%%%%%%%%%%%%%%%%%%%%%

\section*{Course requirements}
\noindent \textbf{REGULAR ATTENDANCE IS ESSENTIAL AND EXPECTED}! 

%%%

\subsection*{Grading criteria, scale, and standards}

\noindent Grades will be based on attendance and a
sequence of approximately 10 labs that may involve
short-answer, essay, mathematical derivations, computer programming,
data analysis, and paper summary/discussion.

\begin{itemize}
\item Reports must be prepared using some form of \TeX language (e.g., Overleaf).
\item Reports will be prepared by rotating pair groups.
\item By the end of the day that reports are due, each group member
  must email me a brief description of their collaborator's strengths
  (why this person is a good collaborator). These reports will be
  compiled and used to determine a collaborator-score that will
  account for 20\% of the course grade. These emails will be kept
  anonymous.
\item The \texttt{R} software will be used for all statistical
  analysis and code/functions should be attached to the report as an
  appendix.
\end{itemize}

\subsection*{Grading Policy:} Labs (80\%),
participation and group collaboration (20\%).

\begin{center}
  \begin{tabular}{ lllll }
    89.5\% & $<$ & A  &      &\\ 
    87\%   & $<$ & B+ &$\leq$& 89.5\%\\ 
    79.5\% & $<$ & B  &$\leq$& 87\%\\ 
    77\%   & $<$ & C+ &$\leq$& 79.5\%\\ 
    69.5\% & $<$ & C  &$\leq$& 77\%\\ 
    67\%   & $<$ & D+ &$\leq$& 69.5\%\\ 
    59.5\% & $<$ & D  &$\leq$& 67\%\\
           &     & F  &$\leq$& 57\%\\
  \end{tabular}
\end{center}

\subsection*{Late  Work Policy:}
Prior consent for rare circumstances or 20\% per day, up to 4 days. 


\section*{Classroom Civility}
To create and preserve a classroom atmosphere that optimizes teaching and learning, all students share the responsibility of creating a positive learning environment. Students are expected to conduct themselves in a manner that does not disrupt teaching or learning, and they are expected to follow these standards:
\begin{itemize}
\item First and foremost, classroom discussion should be civilized, inclusive, and respectful to everyone! Disagreement is expected. Intolerance will not be permitted!
\item You are expected to be on time. I intend to begin class promptly at the designated time, and you should be in your seat and ready to begin class at this time. Class ends at the designated time. Please refrain from packing up your belongings early. It is disruptive to me and, more importantly, to your fellow students.
\item Classroom participation is a part of your grade in this course. You must attend class prepared to fully participate. 
\item Classroom discussion is meant to allow us to hear a variety of viewpoints. This can only happen if we respect each other and our differences.
\item Cell phones must be turned off during class, unless you have informed me ahead of time that you are expecting an emergency message.
\item Students that engage in high-level distractions will be asked to leave class. 
\end{itemize}


\section*{University Policies}

\subsection*{Statement on academic dishonesty:}
The University Academic Standards Policy defines academic dishonesty,
and mandates specific sanctions for violations. See the University
Academic Standards policy: UAM 6,502.

\subsection*{Statement on disability services:}
Any student with a disability needing academic adjustments or
accommodations is requested to speak with me or the Disability
Resource Center (Pennington Achievement Center Suite 230) as soon as
possible to arrange for appropriate accommodations.

\textbf{This course may leverage 3rd party web/multimedia content, if
  you experience any issues accessing this content, please notify your
  instructor}

\subsection*{Statement on audio and video recording:}
\subsection*{Student-created Recordings}
Surreptitious or covert video-taping of class or unauthorized audio recording of class is prohibited by law and by Board of Regents policy. This class may be videotaped, or audio recorded only with the written permission of the instructor. In order to accommodate students with disabilities, some students may have been given permission to record class lectures and discussions. Therefore, students should understand that their comments during class may be recorded.

\subsection*{Instructor-created Recordings}
Class sessions may be audio-visually recorded for students in the class to review and for enrolled students who are unable to attend live to view. Students who participate with their camera on or who use a profile image are consenting to have their video or image recorded. If you do not consent to have your profile or video image recorded, keep your camera off and do not use a profile image. Students who un-mute during class and participate orally are consenting to have their voices recorded. If you do not consent to have your voice recorded during class, keep your mute button activated and only communicate by using the ``chat'' feature, which allows you to type questions and comments live.

\section*{Statement for Academic Success Services}
Your student fees cover usage of the University Math Center [(775) 784-4433], University Tutoring Center [(775) 784-6801], and University Writing Center [(775) 784-6030]. These centers support your classroom learning; it is your responsibility to take advantage of their services.

\section*{Statement on Maintaining a Safe Learning and Work Environment}
The University of Nevada, Reno is committed to providing a safe learning and work environment for all. If you believe you have experienced discrimination, sexual harassment, sexual assault, domestic/dating violence, or stalking, whether on or off campus, or need information related to immigration concerns, please contact the University’s Equal Opportunity \& Title IX office at 775-784-1547. Resources and interim measures are available to assist you. For more information, please visit the Equal Opportunity and Title IX page.

\section*{Statement on COVID-19 Policies}
\subsection*{Face Coverings}
Pursuant to Nevada law, NSHE employees, students and members of the public are not required to wear face coverings while inside NSHE buildings irrespective of vaccination status. However, students may elect to wear face coverings if they choose.

\subsection*{Disinfecting Your Learning Space}
Disinfecting supplies are provided for your convenience to disinfect your learning space. You may also use your own disinfecting supplies.

\subsection*{Testing Positive for COVID-19 or Exhibiting COVID-19 Symptoms}
Students testing positive for COVID-19 or exhibiting COVID-19 symptoms will not be allowed to attend in-person instructional activities and must leave the venue immediately. Students should contact the Student Health Center or their health care provider to receive care and information pertaining to the latest COVID-19 quarantine and self-isolation protocols. If you are required to quarantine or self-isolate, you must contact your instructor immediately to make instructional and learning arrangements.

\subsection*{Accommodations for COVID-19 Quarantined Students}
For students who are required to quarantine or self-isolate due to testing positive for COVID-19 or exhibiting COVID-19 symptoms, instructors must provide opportunities to make-up missed course work, including assignments, quizzes, or exams. In courses with mandatory attendance policies, instructors shall not penalize students for missing classes while quarantined.


\section*{Additional Information}
\subsection*{Methods for communication}
In the event of class cancellation, new information on meeting times,
or room changes, I will send an announcement using WebCampus. Please
email me directly with questions/concerns (perryw@unr.edu). Although I
send announcements to the class using WebCampus, I check my messages
there much less frequently than my UNR email account.

\subsection*{Additional detail about academic dishonesty}
Presenting work that is not your own on exams will result in losing
all points associated with the exam and the inability to make up those
points.


\subsection*{Equal opporunity and Title IX}
The University of Nevada, Reno is committed to providing a safe learning and work environment for all. If you believe you have experienced discrimination, sexual harassment, sexual assault, domestic/dating violence, or stalking, whether on or off campus, or need information related to immigration concerns, please contact the University's Equal Opportunity \& Title IX office at 775-784-1547. Resources and interim measures are available to assist you. For more information, please visit: https://www.unr.edu/equal-opportunity-title-ix .


\section*{Unique class procedures/ structures:}
\subsection*{Working in groups:} You will be assigned a lab group
with another colleague every week. A team approach to work in
the laboratories allows you to teach each other as well as to learn
from me. It will lighten the work load by allowing you to share
tasks. It is more fun.

\subsection*{EECB Diversity Statement}
We, the EECB faculty and students of the University of Nevada, Reno, condemn all forms of discrimination and recognize that systemic bias is pervasive in academia. EECB fully supports the goals of Black Lives Matter, DiversifySTEM, QueerSTEM, DiversifyEEB, Black Ecologists Matter, and allied groups.

We recognize that systemic anti-Blackness, racism, sexism, xenophobia, and transphobia have contributed substantially to a lack of diversity in EECB fields, and we support all individuals and organizations working towards correcting these injustices. We commit to holding ourselves accountable for recognizing and eliminating systemic discrimination within our unit with regards to race, ethnicity, national origins, immigration status, gender, gender identity, sexual orientation, age, ability, socioeconomic status, and religion. We commit ourselves to identifying and dismantling the barriers to accessibility and full participation in our disciplines because we recognize that the greatest scientific advancements will require the full scope of human creativity and experience. We commit to being an active participant and leader in the substantial review and renewal that is needed in education, research and academic policies related to student and faculty equity. Some solutions are within our immediate reach, while others require developing policies and practices that can only be implemented at higher administrative levels or in partnership with our surrounding community. We commit to the long-term work required to develop and implement these actions on our campus, in our local community, as well as our broader scientific societies.





\end{document}
